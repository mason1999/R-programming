\documentclass[12pt, a4paper]{article}\usepackage[]{graphicx}\usepackage[]{color}
% maxwidth is the original width if it is less than linewidth
% otherwise use linewidth (to make sure the graphics do not exceed the margin)
\makeatletter
\def\maxwidth{ %
  \ifdim\Gin@nat@width>\linewidth
    \linewidth
  \else
    \Gin@nat@width
  \fi
}
\makeatother

\definecolor{fgcolor}{rgb}{0.345, 0.345, 0.345}
\newcommand{\hlnum}[1]{\textcolor[rgb]{0.686,0.059,0.569}{#1}}%
\newcommand{\hlstr}[1]{\textcolor[rgb]{0.192,0.494,0.8}{#1}}%
\newcommand{\hlcom}[1]{\textcolor[rgb]{0.678,0.584,0.686}{\textit{#1}}}%
\newcommand{\hlopt}[1]{\textcolor[rgb]{0,0,0}{#1}}%
\newcommand{\hlstd}[1]{\textcolor[rgb]{0.345,0.345,0.345}{#1}}%
\newcommand{\hlkwa}[1]{\textcolor[rgb]{0.161,0.373,0.58}{\textbf{#1}}}%
\newcommand{\hlkwb}[1]{\textcolor[rgb]{0.69,0.353,0.396}{#1}}%
\newcommand{\hlkwc}[1]{\textcolor[rgb]{0.333,0.667,0.333}{#1}}%
\newcommand{\hlkwd}[1]{\textcolor[rgb]{0.737,0.353,0.396}{\textbf{#1}}}%
\let\hlipl\hlkwb

\usepackage{framed}
\makeatletter
\newenvironment{kframe}{%
 \def\at@end@of@kframe{}%
 \ifinner\ifhmode%
  \def\at@end@of@kframe{\end{minipage}}%
  \begin{minipage}{\columnwidth}%
 \fi\fi%
 \def\FrameCommand##1{\hskip\@totalleftmargin \hskip-\fboxsep
 \colorbox{shadecolor}{##1}\hskip-\fboxsep
     % There is no \\@totalrightmargin, so:
     \hskip-\linewidth \hskip-\@totalleftmargin \hskip\columnwidth}%
 \MakeFramed {\advance\hsize-\width
   \@totalleftmargin\z@ \linewidth\hsize
   \@setminipage}}%
 {\par\unskip\endMakeFramed%
 \at@end@of@kframe}
\makeatother

\definecolor{shadecolor}{rgb}{.97, .97, .97}
\definecolor{messagecolor}{rgb}{0, 0, 0}
\definecolor{warningcolor}{rgb}{1, 0, 1}
\definecolor{errorcolor}{rgb}{1, 0, 0}
\newenvironment{knitrout}{}{} % an empty environment to be redefined in TeX

\usepackage{alltt}
\input{./everything/everything.tex}

% for the mathscr
\usepackage{mathrsfs}
% easier for bold symbols
\newcommand{\bs}[1]{\boldsymbol{#1}}
% partial derivatives
\newcommand{\partiald}[1]{\frac{\delta}{\delta#1}}
% second partial derivative
\newcommand{\partialdtwo}[1]{\frac{\delta^2}{\delta#1^2}}
% for estimators
\newcommand{\wh}[1]{\widehat{#1}}
% for examples
\newcommand{\gb}[1]{\greybox{#1}}
% text over arrow
\usepackage{mathtools}
% command for symbols under symbols
\newcommand{\under}[2]{\mathop{#1}\limits_{#2}}
% cancel to zero
\usepackage{cancel}
% Example: \cancelto{0}{x}

\titleformat{\section}{\normalfont\Large\bfseries}{}{0pt}{}
% for sets and curly letters
\newcommand{\cur}[1]{\mathcal{#1}}
\newcommand{\scr}[1]{\mathscr{#1}}
\IfFileExists{upquote.sty}{\usepackage{upquote}}{}
\begin{document}
% refer to https://yihui.org/knitr/options/#code-decoration for more options


% (1) Remember that we want the default page style
% (2) but for this page we want an empty page style for the title 
% (3) we input our title 
% (4) we call a new page and reset the page counter to 1
\pagestyle{default}
\thispagestyle{empty}
\includegraphics[width=8cm]{./UsydLogo}

\vspace{1cm}


\horline
{\centering\bfseries \Large \textsc{STAT3023} Statiscal Inference

}
\horline

\vspace{3cm}

{\large \centering Lab Week 10

}

{\centering

\vspace{1cm}

Tutor: Wen Dai

SID: 470408326

\vspace{1cm}

School of Mathematics and Statistics

The University of Sydney

\vfill

Semester 2, 2021\newpage

}

\newpage
\setcounter{page}{1}

Suppose $\bs{X} = \brac{X_1 ,... , X n}$  is a vector of iid RVs with common PDF $f_\theta\brac{.}$ where:

$$f_\theta\brac{x} = \frac{1}{\theta}g\brac{\frac{x}{\theta}}$$

for a known PDF $g\brac{.}$ which possesses a continuous derivative. The family $\cur{F} = \cbrac{f_\theta\brac{.} : \theta > 0}$ is thus a \textit{scale family} and $\theta$ is a \textit{scale parameter}, like the standard deviation in the normal family. The Cramer-Rao Lower Bound for variance of an unbiased estimator of $\theta$ in such a family based on $n$ iid observations is
given by:

\begin{align*}
\frac{1}{n I_\theta} \text{ where } I_\theta  = \frac{J\brac{g} - 1}{\theta^2} \text{ and } J\brac{g} = \int \frac{\rbrac{xg'\brac{x}}^2}{g\brac{x}}\dx \label{1}\tag{1}
\end{align*}

We shall study what happens when 

$$ g\brac{x} = \frac{1}{\pi\brac{1 + x^2}}$$

is the Cauchy density (same as Student’s-t with 1 degree of freedom, is also the density of the ratio of two independent $N(0, 1)$ random variables); note that the quartiles of $g\brac{.}$ are $\pm 1$, and also that neither the mean nor the variance exist! We shall consider two estimators of $\theta$ based on $\bs{X}$:

\begin{itemize}
  \item $\wh{\theta}_{\text{IQR}}\brac{\bs{X}} = \frac{\text{IQR}\brac{\bs{X}}}{2}$
  \item $\wh{\theta}_{\text{MLE}}\brac{\bs{X}}$, the maximum likelihood estimator (obtained numerically using $R$)
\end{itemize}

\begin{enumerate}[label={\bfseries \arabic*.}]
  \item 
\end{enumerate}
\end{document}
