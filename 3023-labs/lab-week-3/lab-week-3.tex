\documentclass[12pt, a4paper]{article}\usepackage[]{graphicx}\usepackage[]{color}
% maxwidth is the original width if it is less than linewidth
% otherwise use linewidth (to make sure the graphics do not exceed the margin)
\makeatletter
\def\maxwidth{ %
  \ifdim\Gin@nat@width>\linewidth
    \linewidth
  \else
    \Gin@nat@width
  \fi
}
\makeatother

\definecolor{fgcolor}{rgb}{0.345, 0.345, 0.345}
\newcommand{\hlnum}[1]{\textcolor[rgb]{0.686,0.059,0.569}{#1}}%
\newcommand{\hlstr}[1]{\textcolor[rgb]{0.192,0.494,0.8}{#1}}%
\newcommand{\hlcom}[1]{\textcolor[rgb]{0.678,0.584,0.686}{\textit{#1}}}%
\newcommand{\hlopt}[1]{\textcolor[rgb]{0,0,0}{#1}}%
\newcommand{\hlstd}[1]{\textcolor[rgb]{0.345,0.345,0.345}{#1}}%
\newcommand{\hlkwa}[1]{\textcolor[rgb]{0.161,0.373,0.58}{\textbf{#1}}}%
\newcommand{\hlkwb}[1]{\textcolor[rgb]{0.69,0.353,0.396}{#1}}%
\newcommand{\hlkwc}[1]{\textcolor[rgb]{0.333,0.667,0.333}{#1}}%
\newcommand{\hlkwd}[1]{\textcolor[rgb]{0.737,0.353,0.396}{\textbf{#1}}}%
\let\hlipl\hlkwb

\usepackage{framed}
\makeatletter
\newenvironment{kframe}{%
 \def\at@end@of@kframe{}%
 \ifinner\ifhmode%
  \def\at@end@of@kframe{\end{minipage}}%
  \begin{minipage}{\columnwidth}%
 \fi\fi%
 \def\FrameCommand##1{\hskip\@totalleftmargin \hskip-\fboxsep
 \colorbox{shadecolor}{##1}\hskip-\fboxsep
     % There is no \\@totalrightmargin, so:
     \hskip-\linewidth \hskip-\@totalleftmargin \hskip\columnwidth}%
 \MakeFramed {\advance\hsize-\width
   \@totalleftmargin\z@ \linewidth\hsize
   \@setminipage}}%
 {\par\unskip\endMakeFramed%
 \at@end@of@kframe}
\makeatother

\definecolor{shadecolor}{rgb}{.97, .97, .97}
\definecolor{messagecolor}{rgb}{0, 0, 0}
\definecolor{warningcolor}{rgb}{1, 0, 1}
\definecolor{errorcolor}{rgb}{1, 0, 0}
\newenvironment{knitrout}{}{} % an empty environment to be redefined in TeX

\usepackage{alltt}
\input{./everything/everything.tex}

\makeatletter
\def\convertto#1#2{\strip@pt\dimexpr #2*65536/\number\dimexpr 1#1}
\makeatother
\IfFileExists{upquote.sty}{\usepackage{upquote}}{}
\begin{document}

% refer to https://yihui.org/knitr/options/#code-decoration for more options


% (1) Remember that we want the default page style
% (2) but for this page we want an empty page style for the title 
% (3) we input our title 
% (4) we call a new page and reset the page counter to 1
\pagestyle{default}
\thispagestyle{empty}
\includegraphics[width=8cm]{./UsydLogo}

\vspace{1cm}


\horline
{\centering\bfseries \Large \textsc{STAT3023} Statiscal Inference

}
\horline

\vspace{3cm}

{\large \centering Lab Week 10

}

{\centering

\vspace{1cm}

Tutor: Wen Dai

SID: 470408326

\vspace{1cm}

School of Mathematics and Statistics

The University of Sydney

\vfill

Semester 2, 2021\newpage

}

\newpage
\setcounter{page}{1}

% Numerical list
\begin{enumerate}
  \item 
    % alphabetised list
    \begin{enumerate}
      \item Generate 100 realisations of the sample variance of 10 independent $N(0, 1)$ random variables and store them in \verb+s2+
      
      {\setlength{\leftskip}{3ex}
      
      \tbf{Solution}
      
      To do this, we have to recall that sample variance is given by:$$S^2 = \frac{1}{n - 1} \sum_{i = 1}^2(X_i -\overline{X})^2$$We can use the R functin \verb+var+ to obtain the sample variance from a vector
\begin{knitrout}\scriptsize
\definecolor{shadecolor}{rgb}{0.969, 0.969, 0.969}\color{fgcolor}\begin{kframe}
\begin{alltt}
\hlkwd{set.seed}\hlstd{(}\hlnum{3023}\hlstd{)}
\hlstd{num_sim} \hlkwb{=} \hlnum{100}
\hlstd{num_rvs} \hlkwb{=} \hlnum{10}
\hlstd{experiment} \hlkwb{=} \hlkwd{rnorm}\hlstd{(}\hlkwc{n} \hlstd{= num_sim} \hlopt{*} \hlstd{num_rvs,} \hlkwc{mean} \hlstd{=} \hlnum{0}\hlstd{,} \hlkwc{sd} \hlstd{=} \hlnum{1}\hlstd{)}
\hlstd{data_mat} \hlkwb{=} \hlkwd{matrix}\hlstd{(experiment,} \hlkwc{nrow} \hlstd{= num_sim)}
\hlstd{s2} \hlkwb{=} \hlkwd{apply}\hlstd{(data_mat,} \hlkwc{MARGIN} \hlstd{=} \hlnum{1}\hlstd{,} \hlkwc{FUN} \hlstd{= var)}
\end{alltt}
\end{kframe}
\end{knitrout}

      }
      \item Plot the histogram of \verb+(10 - 1) * s2+ and overlay it with the density function of the $\chi_9^2$ distribution (use \verb+dchisq+)
      
      {\setlength{\leftskip}{3ex}
      
      \tbf{Solution}
      
\begin{knitrout}\scriptsize
\definecolor{shadecolor}{rgb}{0.969, 0.969, 0.969}\color{fgcolor}\begin{kframe}
\begin{alltt}
\hlkwd{hist}\hlstd{(s2} \hlopt{*} \hlstd{(num_rvs} \hlopt{-} \hlnum{1}\hlstd{),}
     \hlkwc{main} \hlstd{=} \hlkwd{sprintf}\hlstd{(}\hlstr{"Histogram of s^2 * (%d - 1)"}\hlstd{, num_rvs),}
     \hlkwc{probability} \hlstd{=} \hlnum{TRUE}\hlstd{,}
     \hlkwc{xlab} \hlstd{=} \hlkwd{sprintf}\hlstd{(}\hlstr{"s^2 * (%d - 1)"}\hlstd{, num_rvs))}
\hlkwd{curve}\hlstd{(}\hlkwd{dchisq}\hlstd{(x,} \hlkwc{df} \hlstd{= num_rvs} \hlopt{-} \hlnum{1}\hlstd{),} \hlkwc{add} \hlstd{=} \hlnum{TRUE}\hlstd{)}
\end{alltt}
\end{kframe}

{\centering \includegraphics[width=\maxwidth]{figure/unnamed-chunk-3-1} 

}


\end{knitrout}
      
      }
      
      \item Repeat (a) and (b) with $n = 60$ independent $N(0, 1)$ random variables. Overlay the histogram with both the density curve of $\chi_{n - 1}^2$ and the density curve of $N(n - 1, 2(n - 1))$ (in two different colours). Comment on the fit. 
      
      {\setlength{\leftskip}{3ex}
      
      \tbf{Solution}
      
\begin{knitrout}\scriptsize
\definecolor{shadecolor}{rgb}{0.969, 0.969, 0.969}\color{fgcolor}\begin{kframe}
\begin{alltt}
\hlkwd{set.seed}\hlstd{(}\hlnum{3023}\hlstd{)}
\hlstd{num_sim} \hlkwb{=} \hlnum{100}
\hlstd{num_rvs} \hlkwb{=} \hlnum{60}
\hlstd{experiment} \hlkwb{=} \hlkwd{rnorm}\hlstd{(}\hlkwc{n} \hlstd{= num_sim} \hlopt{*} \hlstd{num_rvs,} \hlkwc{mean} \hlstd{=} \hlnum{0}\hlstd{,} \hlkwc{sd} \hlstd{=} \hlnum{1}\hlstd{)}
\hlstd{data_mat} \hlkwb{=} \hlkwd{matrix}\hlstd{(experiment,} \hlkwc{nrow} \hlstd{= num_sim)}
\hlstd{s2} \hlkwb{=} \hlkwd{apply}\hlstd{(data_mat,} \hlkwc{MARGIN} \hlstd{=} \hlnum{1}\hlstd{,} \hlkwc{FUN} \hlstd{= var)}

\hlkwd{hist}\hlstd{(s2} \hlopt{*} \hlstd{(num_rvs} \hlopt{-} \hlnum{1}\hlstd{),}
     \hlkwc{main} \hlstd{=} \hlkwd{sprintf}\hlstd{(}\hlstr{"Histogram of s^2 * (n - 1)"}\hlstd{),}
     \hlkwc{probability} \hlstd{=} \hlnum{TRUE}\hlstd{,}
     \hlkwc{xlab} \hlstd{=} \hlkwd{sprintf}\hlstd{(}\hlstr{"s^2 * (%d - 1)"}\hlstd{, num_rvs),}
     \hlkwc{breaks} \hlstd{=} \hlnum{15}\hlstd{)}
\hlkwd{curve}\hlstd{(}\hlkwd{dchisq}\hlstd{(x,} \hlkwc{df} \hlstd{= num_rvs} \hlopt{-} \hlnum{1}\hlstd{),}
      \hlkwc{add} \hlstd{=} \hlnum{TRUE}\hlstd{,}
      \hlkwc{col} \hlstd{=} \hlstr{"blue"}\hlstd{)}
\hlkwd{curve}\hlstd{(}\hlkwd{dnorm}\hlstd{(x,} \hlkwc{mean} \hlstd{= num_rvs} \hlopt{-} \hlnum{1}\hlstd{,} \hlkwc{sd} \hlstd{=} \hlkwd{sqrt}\hlstd{(}\hlnum{2}\hlopt{*}\hlstd{(num_rvs} \hlopt{-} \hlnum{1}\hlstd{))),}
      \hlkwc{add} \hlstd{=} \hlnum{TRUE}\hlstd{,}
      \hlkwc{col} \hlstd{=} \hlstr{"red"}\hlstd{)}
\hlkwd{legend}\hlstd{(}\hlkwc{x} \hlstd{=} \hlstr{"topright"}\hlstd{,}
       \hlkwc{legend} \hlstd{=} \hlkwd{c}\hlstd{(}\hlkwd{expression}\hlstd{(chi[n}\hlopt{-}\hlnum{1}\hlstd{]}\hlopt{^}\hlnum{2}\hlstd{),} \hlkwd{expression}\hlstd{(}\hlkwd{N}\hlstd{(n}\hlopt{-}\hlnum{1}\hlstd{,} \hlnum{2}\hlstd{(n}\hlopt{-}\hlnum{1}\hlstd{)))),}
       \hlkwc{col} \hlstd{=} \hlkwd{c}\hlstd{(}\hlstr{"red"}\hlstd{,} \hlstr{"blue"}\hlstd{),}
       \hlkwc{lty} \hlstd{=} \hlkwd{c}\hlstd{(}\hlnum{1}\hlstd{,}\hlnum{1}\hlstd{),}
       \hlkwc{cex} \hlstd{=} \hlnum{0.8}\hlstd{)}
\end{alltt}
\end{kframe}

{\centering \includegraphics[width=\maxwidth]{figure/unnamed-chunk-4-1} 

}


\end{knitrout}
      }
      
      \item For $n = 60$ compute $P\rbrac{\brac{n - 1}S^2 > 68}$ using both the exact distribution $\brac{\chi_{n-1}^2}$ and the normal approximation. Compare the results.
      
      {\setlength{\leftskip}{3ex}
      
      \tbf{Solution}
      
      Recall that $$\frac{\brac{n - 1}S^2}{\sigma^2} = \frac{\sum_{i = 1}^n \brac{X_i - \overline{X}}^2}{\sigma^2} \thicksim \chi_{n-1}^2$$
      And in this case, $\sigma = 1$ and $\mu = 0$. Hence we have:
      $$(n-1)S^2 = \sum_{i = 1}^n\brac{X_i - \overline{X}}^2 \thicksim \chi_{n - 1}^2$$
      
      Thus to compute the $P\rbrac{\brac{n - 1}S^2 > 68}$ we can use the $\chi_{n-1}^2$ distribution or approximate it with $X \thicksim N(n-1, 2(n-1))$
      
\begin{knitrout}\scriptsize
\definecolor{shadecolor}{rgb}{0.969, 0.969, 0.969}\color{fgcolor}\begin{kframe}
\begin{alltt}
\hlkwd{pchisq}\hlstd{(}\hlnum{68}\hlstd{,} \hlkwc{df} \hlstd{= num_rvs} \hlopt{-} \hlnum{1}\hlstd{,} \hlkwc{lower.tail} \hlstd{=} \hlnum{FALSE}\hlstd{)}
\end{alltt}
\begin{verbatim}
[1] 0.1975349
\end{verbatim}
\begin{alltt}
\hlkwd{pnorm}\hlstd{(}\hlnum{68}\hlstd{,} \hlkwc{mean} \hlstd{= num_rvs} \hlopt{-} \hlnum{1}\hlstd{,} \hlkwc{sd} \hlstd{=} \hlkwd{sqrt}\hlstd{(}\hlnum{2}\hlopt{*}\hlstd{(num_rvs} \hlopt{-} \hlnum{1}\hlstd{)),} \hlkwc{lower.tail} \hlstd{=} \hlnum{FALSE}\hlstd{)}
\end{alltt}
\begin{verbatim}
[1] 0.2036888
\end{verbatim}
\end{kframe}
\end{knitrout}

      }
    \end{enumerate}
    \item 
    \begin{enumerate}
    \item When two random variables $(X, Y)$ follow a bivariate normal distribution, the
covariance matrix $\Sigma$ is defined as
$$ \Sigma = \begin{bmatrix} \sigma_1^2 & \rho\sigma_1\sigma_2 \\
\rho\sigma_1\sigma_2 & \sigma_2^2\end{bmatrix}$$
Where $\rho$ is the correlation, $\sigma_1, \sigma_2$ are the variances of $X$ and $Y$ respectively. Use \verb+mvrnorm+ from the \verb+MASS+ library to generate 100 samples
from a bivariate normal distribution with $\boldsymbol{\mu} = \brac{\mu_1, \mu_2}$ with $\mu_1 = 2, \mu_2 = 3$ and $\Sigma = \begin{bmatrix} 1 & 1\\1 & 4\end{bmatrix}$. Call the first column \verb+x+ and the second column \verb+y+
\begin{knitrout}\scriptsize
\definecolor{shadecolor}{rgb}{0.969, 0.969, 0.969}\color{fgcolor}\begin{kframe}
\begin{alltt}
\hlkwd{library}\hlstd{(MASS)}
\hlkwd{set.seed}\hlstd{(}\hlnum{3023}\hlstd{)}
\hlstd{cov_matrix} \hlkwb{=} \hlkwd{matrix}\hlstd{(}\hlkwd{c}\hlstd{(}\hlnum{1}\hlstd{,} \hlnum{1}\hlstd{,} \hlnum{1}\hlstd{,} \hlnum{4}\hlstd{),} \hlkwc{nrow} \hlstd{=} \hlnum{2}\hlstd{,} \hlkwc{ncol} \hlstd{=} \hlnum{2}\hlstd{)}
\hlstd{cov_matrix}
\end{alltt}
\begin{verbatim}
     [,1] [,2]
[1,]    1    1
[2,]    1    4
\end{verbatim}
\begin{alltt}
\hlstd{xy} \hlkwb{=} \hlkwd{mvrnorm}\hlstd{(}\hlnum{100}\hlstd{,} \hlkwc{mu} \hlstd{=} \hlkwd{c}\hlstd{(}\hlnum{2}\hlstd{,} \hlnum{3}\hlstd{),} \hlkwc{Sigma} \hlstd{= cov_matrix)}
\hlstd{x} \hlkwb{=} \hlstd{xy[ ,} \hlnum{1}\hlstd{]}
\hlstd{y} \hlkwb{=} \hlstd{xy[ ,} \hlnum{2}\hlstd{]}
\end{alltt}
\end{kframe}
\end{knitrout}
    \item Plot the histogram of \verb+x+ and overlay it with the corresponding marginal normal
density. Repeat for \verb+y+. (Recall the marginal distribution of $X$ is $N(\mu_1, \sigma_1^2)$)
\begin{knitrout}\scriptsize
\definecolor{shadecolor}{rgb}{0.969, 0.969, 0.969}\color{fgcolor}\begin{kframe}
\begin{alltt}
\hlkwd{par}\hlstd{(}\hlkwc{mfrow} \hlstd{=} \hlkwd{c}\hlstd{(}\hlnum{1}\hlstd{,}\hlnum{2}\hlstd{))}
\hlkwd{hist}\hlstd{(x,} \hlkwc{probability} \hlstd{=} \hlnum{TRUE}\hlstd{,} \hlkwc{main} \hlstd{=} \hlstr{"X"}\hlstd{)}
\hlkwd{curve}\hlstd{(}\hlkwd{dnorm}\hlstd{(x,} \hlkwc{mean} \hlstd{=} \hlnum{2}\hlstd{,} \hlkwc{sd} \hlstd{=} \hlnum{1}\hlstd{),} \hlkwc{add} \hlstd{=} \hlnum{TRUE}\hlstd{,} \hlkwc{col} \hlstd{=} \hlstr{"blue"}\hlstd{)}


\hlkwd{hist}\hlstd{(y,} \hlkwc{probability} \hlstd{=} \hlnum{TRUE}\hlstd{,} \hlkwc{main} \hlstd{=} \hlstr{"Y"}\hlstd{)}
\hlkwd{curve}\hlstd{(}\hlkwd{dnorm}\hlstd{(x,} \hlkwc{mean} \hlstd{=} \hlnum{3}\hlstd{,} \hlkwc{sd} \hlstd{=} \hlnum{2}\hlstd{),} \hlkwc{add} \hlstd{=} \hlnum{TRUE}\hlstd{,} \hlkwc{col} \hlstd{=} \hlstr{"red"}\hlstd{)}
\end{alltt}
\end{kframe}

{\centering \includegraphics[width=\maxwidth]{figure/unnamed-chunk-7-1} 

}


\end{knitrout}
    \item Produce a scatter plot of \verb+x+ and \verb+y+ (use \verb+plot+). Compute the sample correlation coefficient (use \verb+cor+) and compare with the population correlation $\rho$. (First work out $\rho$ in the $\Sigma$ given.)
    
\begin{knitrout}\scriptsize
\definecolor{shadecolor}{rgb}{0.969, 0.969, 0.969}\color{fgcolor}\begin{kframe}
\begin{alltt}
\hlkwd{par}\hlstd{(}\hlkwc{mfrow} \hlstd{=} \hlkwd{c}\hlstd{(}\hlnum{1}\hlstd{,}\hlnum{1}\hlstd{))}
\hlkwd{plot}\hlstd{(x, y)}
\end{alltt}
\end{kframe}

{\centering \includegraphics[width=\maxwidth]{figure/unnamed-chunk-8-1} 

}


\begin{kframe}\begin{alltt}
\hlcom{# Get sample correlation (theoretical correlation is 0.5)}
\hlkwd{cor}\hlstd{(x, y)}
\end{alltt}
\begin{verbatim}
[1] 0.5283183
\end{verbatim}
\end{kframe}
\end{knitrout}
    \end{enumerate}
\item 
  \begin{enumerate}
    \item Generate 100 realizations of the minimum of 10 independent exponential(1) random variables. Note the rate parameter in \verb+rexp+ is defined as the reciprocal the
expectation (check the density function in the help file ?\verb+rexp+).

{\setlength{\leftskip}{3ex}

\tbf{Solution}

We recall that if $X \thicksim exp(\lambda)$ then $\E{X} = \frac{1}{\lambda}$. Hence the \verb+rate+ parameter in R is $\lambda$. 
\begin{knitrout}\scriptsize
\definecolor{shadecolor}{rgb}{0.969, 0.969, 0.969}\color{fgcolor}\begin{kframe}
\begin{alltt}
\hlstd{n} \hlkwb{=} \hlnum{10}
\hlstd{num_sim} \hlkwb{=} \hlnum{100}
\hlstd{lambda} \hlkwb{=} \hlnum{1}
\hlstd{mat_sim} \hlkwb{=} \hlkwd{matrix}\hlstd{(}\hlkwd{rexp}\hlstd{(}\hlkwc{n} \hlstd{= num_sim} \hlopt{*} \hlstd{n,} \hlkwc{rate} \hlstd{= lambda),}
                 \hlkwc{nrow} \hlstd{= num_sim,}
                 \hlkwc{ncol} \hlstd{= n)}
\hlstd{simulate_mins} \hlkwb{=} \hlkwd{apply}\hlstd{(mat_sim,} \hlkwc{MARGIN} \hlstd{=} \hlnum{1}\hlstd{,} \hlkwc{FUN} \hlstd{= min)}
\end{alltt}
\end{kframe}
\end{knitrout}

}
    \item Plot the histogram and overlay it with the density of exponential(1/10) (\verb+rate+=10) distribution. Comment on the fit.
\begin{knitrout}\scriptsize
\definecolor{shadecolor}{rgb}{0.969, 0.969, 0.969}\color{fgcolor}\begin{kframe}
\begin{alltt}
\hlkwd{par}\hlstd{(}\hlkwc{mfrow} \hlstd{=} \hlkwd{c}\hlstd{(}\hlnum{1}\hlstd{,}\hlnum{1}\hlstd{))}
\hlkwd{hist}\hlstd{(}\hlkwc{x} \hlstd{= simulate_mins,} \hlkwc{probability} \hlstd{=} \hlnum{TRUE}\hlstd{,}
     \hlkwc{xlab} \hlstd{=} \hlstr{"minimums"}\hlstd{,}
     \hlkwc{ylab} \hlstd{=} \hlstr{"probability"}\hlstd{)}
\hlkwd{curve}\hlstd{(}\hlkwd{dexp}\hlstd{(x,} \hlkwc{rate} \hlstd{= n),} \hlkwc{add} \hlstd{=} \hlnum{TRUE}\hlstd{)}
\end{alltt}
\end{kframe}

{\centering \includegraphics[width=\maxwidth]{figure/unnamed-chunk-10-1} 

}


\end{knitrout}
  \end{enumerate}
\end{enumerate}
\end{document}
