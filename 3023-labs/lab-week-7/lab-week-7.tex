\documentclass[12pt, a4paper]{article}\usepackage[]{graphicx}\usepackage[]{color}
% maxwidth is the original width if it is less than linewidth
% otherwise use linewidth (to make sure the graphics do not exceed the margin)
\makeatletter
\def\maxwidth{ %
  \ifdim\Gin@nat@width>\linewidth
    \linewidth
  \else
    \Gin@nat@width
  \fi
}
\makeatother

\definecolor{fgcolor}{rgb}{0.345, 0.345, 0.345}
\newcommand{\hlnum}[1]{\textcolor[rgb]{0.686,0.059,0.569}{#1}}%
\newcommand{\hlstr}[1]{\textcolor[rgb]{0.192,0.494,0.8}{#1}}%
\newcommand{\hlcom}[1]{\textcolor[rgb]{0.678,0.584,0.686}{\textit{#1}}}%
\newcommand{\hlopt}[1]{\textcolor[rgb]{0,0,0}{#1}}%
\newcommand{\hlstd}[1]{\textcolor[rgb]{0.345,0.345,0.345}{#1}}%
\newcommand{\hlkwa}[1]{\textcolor[rgb]{0.161,0.373,0.58}{\textbf{#1}}}%
\newcommand{\hlkwb}[1]{\textcolor[rgb]{0.69,0.353,0.396}{#1}}%
\newcommand{\hlkwc}[1]{\textcolor[rgb]{0.333,0.667,0.333}{#1}}%
\newcommand{\hlkwd}[1]{\textcolor[rgb]{0.737,0.353,0.396}{\textbf{#1}}}%
\let\hlipl\hlkwb

\usepackage{framed}
\makeatletter
\newenvironment{kframe}{%
 \def\at@end@of@kframe{}%
 \ifinner\ifhmode%
  \def\at@end@of@kframe{\end{minipage}}%
  \begin{minipage}{\columnwidth}%
 \fi\fi%
 \def\FrameCommand##1{\hskip\@totalleftmargin \hskip-\fboxsep
 \colorbox{shadecolor}{##1}\hskip-\fboxsep
     % There is no \\@totalrightmargin, so:
     \hskip-\linewidth \hskip-\@totalleftmargin \hskip\columnwidth}%
 \MakeFramed {\advance\hsize-\width
   \@totalleftmargin\z@ \linewidth\hsize
   \@setminipage}}%
 {\par\unskip\endMakeFramed%
 \at@end@of@kframe}
\makeatother

\definecolor{shadecolor}{rgb}{.97, .97, .97}
\definecolor{messagecolor}{rgb}{0, 0, 0}
\definecolor{warningcolor}{rgb}{1, 0, 1}
\definecolor{errorcolor}{rgb}{1, 0, 0}
\newenvironment{knitrout}{}{} % an empty environment to be redefined in TeX

\usepackage{alltt}
\input{./everything/everything.tex}
% for the mathscr
\usepackage{mathrsfs}
% easier for bold symbols
\newcommand{\bs}[1]{\boldsymbol{#1}}
% partial derivatives
\newcommand{\partiald}[1]{\frac{\delta}{\delta#1}}
% second partial derivative
\newcommand{\partialdtwo}[1]{\frac{\delta^2}{\delta#1^2}}
% for estimators
\newcommand{\wh}[1]{\widehat{#1}}
% for examples
\newcommand{\gb}[1]{\greybox{#1}}
% text over arrow
\usepackage{mathtools}
% command for symbols under symbols
\newcommand{\under}[2]{\mathop{#1}\limits_{#2}}
% cancel to zero
\usepackage{cancel}
% Example: \cancelto{0}{x}

\titleformat{\section}{\normalfont\Large\bfseries}{}{0pt}{}
% for sets and curly letters
\newcommand{\cur}[1]{\mathcal{#1}}
\newcommand{\scr}[1]{\mathscr{#1}}

\usepackage{listings}
\definecolor{royalblue}{rgb}{0.25, 0.41, 0.88}
\definecolor{forestgreen}{rgb}{0.0, 0.27, 0.13}
\definecolor{rust}{rgb}{0.72, 0.25, 0.05}

\lstset{ 
  language=R,                     % the language of the code
  basicstyle=\footnotesize, % the size of the fonts that are used for the code
  numbers=left,                   % where to put the line-numbers
  numberstyle=\tiny\color{gray!150},  % the style that is used for the line-numbers
  stepnumber=1,                   % the step between two line-numbers. If it is 1, each line
                                  % will be numbered
  numbersep=5pt,                  % how far the line-numbers are from the code
  backgroundcolor=\color{gray!40},  % choose the background color. You must add \usepackage{color}
  showspaces=false,               % show spaces adding particular underscores
  showstringspaces=false,         % underline spaces within strings
  showtabs=false,                 % show tabs within strings adding particular underscores
  frame=single,                   % adds a frame around the code
  rulecolor=\color{black},        % if not set, the frame-color may be changed on line-breaks within not-black text (e.g. commens (green here))
  tabsize=2,                      % sets default tabsize to 2 spaces
  captionpos=b,                   % sets the caption-position to bottom
  breaklines=true,                % sets automatic line breaking
  breakatwhitespace=false,        % sets if automatic breaks should only happen at whitespace
  keywordstyle=\color{royalblue},      % keyword style
  commentstyle=\color{forestgreen},   % comment style
  stringstyle=\color{rust}      % string literal style
} 
\IfFileExists{upquote.sty}{\usepackage{upquote}}{}
\begin{document}
% refer to https://yihui.org/knitr/options/#code-decoration for more options


% (1) Remember that we want the default page style
% (2) but for this page we want an empty page style for the title 
% (3) we input our title 
% (4) we call a new page and reset the page counter to 1
\pagestyle{default}
\thispagestyle{empty}
\includegraphics[width=8cm]{./UsydLogo}

\vspace{1cm}


\horline
{\centering\bfseries \Large \textsc{STAT3023} Statiscal Inference

}
\horline

\vspace{3cm}

{\large \centering Lab Week 10

}

{\centering

\vspace{1cm}

Tutor: Wen Dai

SID: 470408326

\vspace{1cm}

School of Mathematics and Statistics

The University of Sydney

\vfill

Semester 2, 2021\newpage

}

\newpage
\setcounter{page}{1}

Consider the test of $H_0: \theta = \theta_0$ against $H_1 : \theta \neq \theta_0$ based on $X\thicksim f(. ; \gamma_0, \theta)$ where
\begin{align}
f\brac{x; \gamma, \theta} = 
\begin{cases}
\frac{1}{\Gamma\brac{\gamma}\theta^\gamma}x^{\gamma - 1}e^{-\frac{x}{\theta}} &\quad\text{for } x > 0 \\
0 &\quad\text{otherwise}
\end{cases}
\end{align}
that is $X$ has a gamma distribution with known shape parameter $\gamma_0$ but unknown scale parameter $\theta$.

\begin{enumerate}[label = {\bfseries \arabic*.}]
\item Consider first the exponential case where $\gamma_0 = 1$ and suppose the hypothesised value of the scale parameter (also the mean in this case) is $\theta_0 = 1$

  \begin{enumerate}[label = (\alph*)]
  \item The ``equal-tailed''test at level $\alpha$ rejects for $X \le a$ or $X \ge b$ where
    \begin{align}
    P_1\brac{X\le a} = P_1\brac{X\ge b} = \frac{\alpha}{2}
    \end{align}
    Taking $\alpha = 0.05$, determine the value sof \verb+a+ and \verb+b+ satisfying (2) above (\tbf{hint:} use \verb+qexp()+).
    
{\setlength{\leftskip}{3ex}
\tbf{Solution}

Since we take $\gamma_0 = 1$ and under the null $\theta_0 = 1$ we have that the the pdf is given by: 
$$ f(x) = e^{-x}\qquad\text{for } x > 0$$
\begin{knitrout}\footnotesize
\definecolor{shadecolor}{rgb}{0.969, 0.969, 0.969}\color{fgcolor}\begin{kframe}
\begin{alltt}
\hlstd{a} \hlkwb{=} \hlkwd{qexp}\hlstd{(}\hlnum{0.025}\hlstd{,} \hlkwc{rate} \hlstd{=} \hlnum{1}\hlstd{)}
\hlstd{b} \hlkwb{=} \hlkwd{qexp}\hlstd{(}\hlnum{0.975}\hlstd{,} \hlkwc{rate} \hlstd{=} \hlnum{1}\hlstd{)}
\hlkwd{c}\hlstd{(a, b)}
\end{alltt}
\begin{verbatim}
## [1] 0.02531781 3.68887945
\end{verbatim}
\end{kframe}
\end{knitrout}
}
    \item We shall plot the power function of the equal-tailed test. Define a vector of $\theta$-values: \verb+th=(250:1500)/1000+ and obtain a corresponding vector of values of the power (the proba- bility of rejecting) for each such $\theta$-value; that is:
    $$P_\theta \brac{X\le a} + P_\theta \brac{ X \ge b}$$
    Finally plot the power against \verb+th+ and add a horizontal dashed line at $\gamma = 0.05$. Add an informative heading, etc and remember to use \verb+type = 'l'+
    
    {\setlength{\leftskip}{3ex}
    
    \tbf{Solution}
    
    We plot the power function of the equal-tailed test. Note that for $P_\theta\brac{X \ge b}$ we note that $P_\theta\brac{X \ge b} = 1 - P_\theta\brac{X \le b}$. We alse use the \verb+abline(.)+ function to plot a horizontal line
    
\begin{knitrout}\footnotesize
\definecolor{shadecolor}{rgb}{0.969, 0.969, 0.969}\color{fgcolor}\begin{kframe}
\begin{alltt}
\hlstd{th} \hlkwb{=} \hlstd{(}\hlnum{250}\hlopt{:}\hlnum{1500}\hlstd{)}\hlopt{/}\hlnum{1000}
\hlstd{power} \hlkwb{=}  \hlkwd{pexp}\hlstd{(a,} \hlkwc{rate} \hlstd{=} \hlnum{1}\hlopt{/}\hlstd{th,} \hlkwc{lower.tail} \hlstd{=} \hlnum{TRUE}\hlstd{)} \hlopt{+} \hlkwd{pexp}\hlstd{(b,} \hlkwc{rate} \hlstd{=} \hlnum{1}\hlopt{/}\hlstd{th,} \hlkwc{lower.tail} \hlstd{=} \hlnum{FALSE}\hlstd{)}
\hlcom{# Plot theta against the power:}
\hlkwd{plot}\hlstd{(th, power,} \hlkwc{type} \hlstd{=} \hlstr{"l"}\hlstd{,} \hlkwc{main} \hlstd{=} \hlstr{"Power of two tail test"}\hlstd{)}
\hlcom{# add a horizontal dashed line at gamma = 0.05}
\hlkwd{abline}\hlstd{(}\hlkwc{h} \hlstd{=} \hlnum{0.05}\hlstd{,} \hlkwc{lty} \hlstd{=} \hlnum{2}\hlstd{)}
\end{alltt}
\end{kframe}

{\centering \includegraphics[width=\maxwidth]{figure/unnamed-chunk-3-1} 

}


\end{knitrout}
    
    }
    \item This is a 1-parameter exponential family with sufficient statistic $X$ and so (since it is continuous) the \textit{uniformly most powerful unbiased} (UMPU) test is of the form
    \begin{align}
    \delta\brac{X} = 
      \begin{cases}
      1 &\qquad\text{for } X \ge d \\
      0 &\qquad\text{for } c < X < d\\
      1 &\qquad \text{for } X \le c
      \end{cases}
    \end{align}
Where $c$ and $d$ are chosen so that:
  \begin{align}
    \E_{\theta_0}\rbrac{\delta\brac{X}} = \alpha
  \end{align}
  \begin{align}
  \E_{\theta_0}\rbrac{X\delta\brac{X}} = \alpha\E_{\theta_0}\brac{X} = \alpha
  \end{align}
  Since $\E_{\theta_0}\brac{X} = 1$. We show in a tutorial exercise these are equivalent to
  \begin{align*}
  1 - e^{-c} + e^{-d} & = \alpha \\
  ce^{-c} & = de^{-d}
  \end{align*}
  From the first equation we get:
  \begin{align*}
  e^{-d} & = \alpha - 1 + e^{-c} \\
  \implies d & = -\log \rbrac{\alpha -1 + e^{-c}}
  \end{align*}
  Hence, once $c$ is determined we can compute $d$. To determine $c$ we need to solve the equation:
  \begin{align*}
  ce^{-c} - de^{-d} = ce^{-c} + \cbrac{\log \rbrac{\alpha - 1 + e^{-c}}}\rbrac{\alpha - 1 + e^{-c}} = 0
  \end{align*}
  We can use the R function \verb+uniroot()+ to determine c \textit{numerically}.
    \begin{enumerate}[label=(\roman*)]
      \item Write an R function which computes the middle member of the equation above (i.e. the function whose root we wish to find).
      
      {\setlength{\leftskip}{3ex}
      \tbf{Solution}
      
      We write a function which computes the root $c$ to the equation 
      $$ce^{-c} + \cbrac{\log \rbrac{\alpha - 1 + e^{-c}}}\rbrac{\alpha - 1 + e^{-c}} = 0$$
\begin{knitrout}\footnotesize
\definecolor{shadecolor}{rgb}{0.969, 0.969, 0.969}\color{fgcolor}\begin{kframe}
\begin{alltt}
\hlstd{fn} \hlkwb{=} \hlkwa{function}\hlstd{(}\hlkwc{c}\hlstd{,} \hlkwc{alpha}\hlstd{) \{}
  \hlstd{term} \hlkwb{=} \hlstd{alpha} \hlopt{-} \hlnum{1} \hlopt{+} \hlkwd{exp}\hlstd{(}\hlopt{-}\hlnum{1} \hlopt{*} \hlstd{c)}
  \hlkwd{return}\hlstd{(c} \hlopt{*} \hlkwd{exp}\hlstd{(}\hlopt{-}\hlstd{c)} \hlopt{+} \hlstd{(}\hlkwd{log}\hlstd{(term)} \hlopt{*} \hlstd{term))}
\hlstd{\}}
\end{alltt}
\end{kframe}
\end{knitrout}
      
      }
      \item Noting that $c$ can be no bigger than the lower 0.05-quantile of the exponential(1) distribution, execute a certain command involving \verb+eps = 1e-5+. Note: the use of \verb+eps+ here is to stay away from the upper bound, since there the function is trying to evaluate \verb+log(0)+
\begin{knitrout}\footnotesize
\definecolor{shadecolor}{rgb}{0.969, 0.969, 0.969}\color{fgcolor}\begin{kframe}
\begin{alltt}
\hlstd{eps} \hlkwb{=} \hlnum{1e-5}
\hlkwd{uniroot}\hlstd{(}\hlkwc{f} \hlstd{= fn,} \hlkwc{lower} \hlstd{=} \hlnum{0}\hlstd{,} \hlkwc{upper} \hlstd{=} \hlkwd{qexp}\hlstd{(}\hlnum{0.05}\hlstd{)} \hlopt{-} \hlstd{eps,} \hlkwc{alpha} \hlstd{=} \hlnum{0.05}\hlstd{)}
\end{alltt}
\begin{verbatim}
## $root
## [1] 0.04235629
## 
## $f.root
## [1] -3.187136e-05
## 
## $iter
## [1] 4
## 
## $init.it
## [1] NA
## 
## $estim.prec
## [1] 6.103516e-05
\end{verbatim}
\end{kframe}
\end{knitrout}
      \item Write an R function which takes as an argument the level \verb+alpha+ and returns a list with elements \verb+c+ and \verb+d+, corresponding to the desired values $c$ and $d$ defining the UMPU test (3) above for $\theta_0 = 1$ and $\alpha = 0.05$
      
      {\setlength{\leftskip}{3ex}
      
      \tbf{Solution}
      
      We write a function \verb+expon.umpu+ (exponential uniformly most powerful unbiased test) which takes in $\alpha$ and returns the optimal $c$ and $d$ which solve the above equations. 
      
\begin{knitrout}\footnotesize
\definecolor{shadecolor}{rgb}{0.969, 0.969, 0.969}\color{fgcolor}\begin{kframe}
\begin{alltt}
\hlstd{expon.umpu} \hlkwb{=} \hlkwa{function}\hlstd{(}\hlkwc{alpha}\hlstd{) \{}
  \hlstd{eps} \hlkwb{=} \hlnum{1e-8}
  \hlstd{c} \hlkwb{=} \hlkwd{uniroot}\hlstd{(}\hlkwc{f} \hlstd{= fn,} \hlkwc{lower} \hlstd{=} \hlnum{0}\hlstd{,} \hlkwc{upper} \hlstd{=} \hlkwd{qexp}\hlstd{(alpha)} \hlopt{-} \hlstd{eps,} \hlkwc{alpha} \hlstd{= alpha)}\hlopt{$}\hlstd{root}
  \hlstd{term} \hlkwb{=} \hlstd{alpha} \hlopt{-} \hlnum{1} \hlopt{+} \hlkwd{exp}\hlstd{(}\hlopt{-}\hlstd{c)}
  \hlstd{d} \hlkwb{=} \hlopt{-}\hlkwd{log}\hlstd{(term)}
  \hlkwd{list}\hlstd{(}\hlkwc{c_val} \hlstd{= c,} \hlkwc{d_val} \hlstd{= d)}
\hlstd{\}}
\hlkwd{expon.umpu}\hlstd{(}\hlnum{0.05}\hlstd{)}
\end{alltt}
\begin{verbatim}
## $c_val
## [1] 0.04235611
## 
## $d_val
## [1] 4.764356
\end{verbatim}
\end{kframe}
\end{knitrout}
      
      }
    \end{enumerate}
  \item Recreate your plot from part (b) above and add to it a red curve of the power of the UMPU test. Add an informative heading and legend. \tbf{Comment} on what feature of the plot indicates that the UMPU test is unbiased. The power function of the UMPU test never goes below the 0.05 level, thus it is unbiased.
  
  {\setlength{\leftskip}{3ex}
  \tbf{Solution}
  
  We now plot the power of the UMPU test in red:
  
\begin{knitrout}\footnotesize
\definecolor{shadecolor}{rgb}{0.969, 0.969, 0.969}\color{fgcolor}\begin{kframe}
\begin{alltt}
\hlstd{th} \hlkwb{=} \hlstd{(}\hlnum{250}\hlopt{:}\hlnum{1500}\hlstd{)}\hlopt{/}\hlnum{1000}
\hlcom{# for the two tailed test}
\hlstd{power} \hlkwb{=} \hlkwd{pexp}\hlstd{(a,} \hlnum{1}\hlopt{/}\hlstd{th)} \hlopt{+} \hlkwd{pexp}\hlstd{(b,} \hlnum{1}\hlopt{/}\hlstd{th,} \hlkwc{lower.tail} \hlstd{=} \hlnum{FALSE}\hlstd{)}
\hlkwd{plot}\hlstd{(th, power,} \hlkwc{type} \hlstd{=} \hlstr{"l"}\hlstd{,} \hlkwc{main} \hlstd{=} \hlstr{"testing for an exponential mean"}\hlstd{)}
\hlkwd{abline}\hlstd{(}\hlkwc{h} \hlstd{=} \hlnum{0.05}\hlstd{,} \hlkwc{lty} \hlstd{=} \hlnum{2}\hlstd{)}
\hlcom{# for the uniformly most powerful test:}
\hlstd{umpu} \hlkwb{=} \hlkwd{expon.umpu}\hlstd{(}\hlnum{0.05}\hlstd{)}
\hlstd{umpu.power} \hlkwb{=} \hlkwd{pexp}\hlstd{(umpu}\hlopt{$}\hlstd{c,} \hlnum{1}\hlopt{/}\hlstd{th)} \hlopt{+} \hlkwd{pexp}\hlstd{(umpu}\hlopt{$}\hlstd{d,} \hlnum{1}\hlopt{/}\hlstd{th,} \hlkwc{lower.tail} \hlstd{=} \hlnum{FALSE}\hlstd{)}
\hlkwd{lines}\hlstd{(th, umpu.power,} \hlkwc{col} \hlstd{=} \hlstr{"red"}\hlstd{)}
\hlkwd{legend}\hlstd{(}\hlstr{"top"}\hlstd{,} \hlkwc{legend} \hlstd{=} \hlkwd{c}\hlstd{(}\hlstr{"Power of equal-tailed test"}\hlstd{,} \hlstr{"power of UMPU test"}\hlstd{),}
       \hlkwc{col} \hlstd{=} \hlkwd{c}\hlstd{(}\hlstr{"black"}\hlstd{,} \hlstr{"red"}\hlstd{),}
       \hlkwc{lty} \hlstd{=} \hlkwd{c}\hlstd{(}\hlnum{1}\hlstd{,} \hlnum{1}\hlstd{))}
\end{alltt}
\end{kframe}

{\centering \includegraphics[width=\maxwidth]{figure/unnamed-chunk-7-1} 

}


\end{knitrout}
  
  }
  \end{enumerate}
\item Consider now the case where $Y_1, ..., Y_n$ are iid exponential with mean $\theta$ and we again wish to test $H_0: \theta = \theta_0$ against a two sided $H_1: \theta \neq \theta_0$. The likelihood is:
$$\Pi_{i = 1}^n\rbrac{\frac{1}{\theta}e^{-\frac{Y_i}{\theta}}} = \exp\brac{-\frac{1}{\theta}\sum_{i = 1}^nY_i - n\log\theta}$$
and so clearly $X = \sum_{i = 1}^nY_i$ is a sufficient statistic; indeed this is a 1-parameter exponential family. The UMPU test is thus of the form (3) where $c$ and $d$ are chosen to satisfy the two conditions (4) and (5).

However since $X$ itself has a gamma distribution with shape parameter $n$ and scale parameter $\theta$, the UMPU above is the same as for a single observation from the density (1) above with $\gamma_0 = n$. We show in the tutorial that in the present case the two conditions (4) and (5) are equivalent to
$$ \int_0^cf\brac{x; n, 1}\dx + \int_d^\infty f\brac{x; n, 1}\dx = \alpha = \int_0^c f\brac{x; n + 1, 1}\dx + \int_d^\infty f\brac{x; n + 1, 1}\dx$$

Below we shall write a function to determine the UMPU test, for the case $n = 5$ and $\alpha = 0.05$.
\begin{enumerate}[label=(\alph*)]
  \item By adapting your solution to part (c) of the previous question, write a function (playing the same role as the function \verb+fn()+ above; it will use \verb+pgamma()+ and \verb+qgamma()+) the root of which gives the desired value of $c$ to solve the above equations (\tbf{hint}: in the body of this function you will need to first find \verb+d+ in terms of \verb+c+ and \verb+alpha+ using one of the two constraints). Then use \verb+uniroot()+ to actually find the root. Wrap this all in an appropriate function which takes as input values of \verb+alpha+ and \verb+n+ and outputs a list with elements \verb+c+ and \verb+d+, the lower and upper critical values of the desired UMPU test.
  
  {\setlength{\leftskip}{3ex}
  \tbf{Solution}
  
\begin{knitrout}\footnotesize
\definecolor{shadecolor}{rgb}{0.969, 0.969, 0.969}\color{fgcolor}\begin{kframe}
\begin{alltt}
\hlstd{gamma.root} \hlkwb{=} \hlkwa{function}\hlstd{(}\hlkwc{c}\hlstd{,} \hlkwc{n}\hlstd{,} \hlkwc{alpha}\hlstd{) \{}
  \hlstd{lower} \hlkwb{=} \hlkwd{pgamma}\hlstd{(}\hlkwc{q} \hlstd{= c,} \hlkwc{shape} \hlstd{= n,} \hlkwc{scale} \hlstd{=} \hlnum{1}\hlstd{)} \hlcom{# P (X < c)}
  \hlcom{# P(X < c) + P(X > d) = alpha}
  \hlcom{# P(X > d) = alpha - P(x < c)}
  \hlcom{# P(X < d) = 1 - alpha + P(X < c) }
  \hlcom{# d = F^-1(1 - alpha + P(X < c))}
  \hlstd{d} \hlkwb{=} \hlkwd{qgamma}\hlstd{(}\hlkwc{p} \hlstd{=} \hlnum{1} \hlopt{-} \hlstd{alpha} \hlopt{+} \hlstd{lower,} \hlkwc{shape} \hlstd{= n,} \hlkwc{scale} \hlstd{=} \hlnum{1}\hlstd{)}

  \hlcom{# return the equation to solve--> the latter half of the integral equation}
  \hlkwd{return}\hlstd{(}\hlkwd{pgamma}\hlstd{(c,} \hlkwc{shape} \hlstd{= (n} \hlopt{+} \hlnum{1}\hlstd{),} \hlkwc{scale} \hlstd{=} \hlnum{1}\hlstd{)} \hlopt{+}
           \hlkwd{pgamma}\hlstd{(d,} \hlkwc{shape} \hlstd{= (n} \hlopt{+} \hlnum{1}\hlstd{),} \hlkwc{scale} \hlstd{=} \hlnum{1}\hlstd{,} \hlkwc{lower.tail} \hlstd{=} \hlnum{FALSE}\hlstd{)} \hlopt{-}
           \hlstd{alpha)}
\hlstd{\}}
\end{alltt}
\end{kframe}
\end{knitrout}
Show the roots:
\begin{knitrout}\footnotesize
\definecolor{shadecolor}{rgb}{0.969, 0.969, 0.969}\color{fgcolor}\begin{kframe}
\begin{alltt}
\hlstd{eps} \hlkwb{=} \hlnum{1e-8}
\hlstd{upper} \hlkwb{=} \hlkwd{qgamma}\hlstd{(}\hlnum{0.05}\hlstd{,} \hlkwc{shape} \hlstd{=} \hlnum{5}\hlstd{,} \hlkwc{scale} \hlstd{=} \hlnum{1}\hlstd{)} \hlopt{-} \hlstd{eps}
\hlkwd{uniroot}\hlstd{(}\hlkwc{f} \hlstd{= gamma.root,} \hlkwc{lower} \hlstd{=} \hlnum{0}\hlstd{,} \hlkwc{upper} \hlstd{= upper,} \hlkwc{n} \hlstd{=} \hlnum{5}\hlstd{,} \hlkwc{alpha} \hlstd{=} \hlnum{0.05}\hlstd{)}
\end{alltt}
\begin{verbatim}
## $root
## [1] 1.758069
## 
## $f.root
## [1] 1.352949e-06
## 
## $iter
## [1] 6
## 
## $init.it
## [1] NA
## 
## $estim.prec
## [1] 6.103516e-05
\end{verbatim}
\end{kframe}
\end{knitrout}
And finally wrap all this up in a function:
\begin{knitrout}\footnotesize
\definecolor{shadecolor}{rgb}{0.969, 0.969, 0.969}\color{fgcolor}\begin{kframe}
\begin{alltt}
\hlstd{gamma.umpu} \hlkwb{=} \hlkwa{function}\hlstd{(}\hlkwc{alpha}\hlstd{,} \hlkwc{n}\hlstd{) \{}
  \hlstd{eps} \hlkwb{=} \hlnum{1e-8}
  \hlstd{upper} \hlkwb{=} \hlkwd{qgamma}\hlstd{(alpha,} \hlkwc{shape} \hlstd{= n,} \hlkwc{scale} \hlstd{=} \hlnum{1}\hlstd{)} \hlopt{-} \hlstd{eps}
  \hlstd{c} \hlkwb{=} \hlkwd{uniroot}\hlstd{(}\hlkwc{f} \hlstd{= gamma.root,} \hlkwc{lower} \hlstd{=} \hlnum{0}\hlstd{,} \hlkwc{upper} \hlstd{= upper,} \hlkwc{n} \hlstd{= n,} \hlkwc{alpha} \hlstd{= alpha)}\hlopt{$}\hlstd{root}
  \hlstd{lower} \hlkwb{=} \hlkwd{pgamma}\hlstd{(c,} \hlkwc{shape} \hlstd{= n,} \hlkwc{scale} \hlstd{=} \hlnum{1}\hlstd{)}
  \hlstd{d} \hlkwb{=} \hlkwd{qgamma}\hlstd{(}\hlnum{1} \hlopt{-} \hlstd{(alpha} \hlopt{-} \hlstd{lower),} \hlkwc{shape} \hlstd{= n,} \hlkwc{scale} \hlstd{=} \hlnum{1}\hlstd{)}
  \hlkwd{return}\hlstd{(}\hlkwd{list}\hlstd{(}\hlkwc{c_val} \hlstd{= c,} \hlkwc{d_val} \hlstd{= d))}
\hlstd{\}}
\end{alltt}
\end{kframe}
\end{knitrout}
  
  }
  \item Use your \verb+gamma.umpu()+ function to determine the appropriate $c$ and $d$ for the UMPU test for this problem with $n = 5$ and $\alpha = 0.05$. Plot the power as a function of $\theta$ and graphically verify that the test is unbiased and of level 0.05.
\begin{knitrout}\footnotesize
\definecolor{shadecolor}{rgb}{0.969, 0.969, 0.969}\color{fgcolor}\begin{kframe}
\begin{alltt}
\hlkwd{gamma.umpu}\hlstd{(}\hlnum{0.05}\hlstd{,} \hlnum{5}\hlstd{)}
\end{alltt}
\begin{verbatim}
## $c_val
## [1] 1.758069
## 
## $d_val
## [1] 10.86438
\end{verbatim}
\begin{alltt}
\hlstd{gu} \hlkwb{=} \hlkwd{gamma.umpu}\hlstd{(}\hlnum{0.05}\hlstd{,} \hlnum{5}\hlstd{)} \hlcom{# obtain c and d}
\hlstd{g.power} \hlkwb{=} \hlkwd{pgamma}\hlstd{(gu}\hlopt{$}\hlstd{c,} \hlkwc{shape} \hlstd{=} \hlnum{5}\hlstd{,} \hlkwc{scale} \hlstd{= th)} \hlopt{+}
  \hlnum{1} \hlopt{-} \hlkwd{pgamma}\hlstd{(gu}\hlopt{$}\hlstd{d,} \hlkwc{shape} \hlstd{=} \hlnum{5}\hlstd{,} \hlkwc{scale} \hlstd{= th)}
\hlkwd{plot}\hlstd{(th, g.power,} \hlkwc{type} \hlstd{=} \hlstr{"l"}\hlstd{)}
\hlkwd{abline}\hlstd{(}\hlkwc{h} \hlstd{=} \hlnum{0.05}\hlstd{,} \hlkwc{lty} \hlstd{=} \hlnum{2}\hlstd{)}
\end{alltt}
\end{kframe}

{\centering \includegraphics[width=\maxwidth]{figure/unnamed-chunk-11-1} 

}


\end{knitrout}
\end{enumerate}
\end{enumerate}

\newpage

\end{document}
