\documentclass[12pt, a4paper]{article}\usepackage[]{graphicx}\usepackage[]{color}
% maxwidth is the original width if it is less than linewidth
% otherwise use linewidth (to make sure the graphics do not exceed the margin)
\makeatletter
\def\maxwidth{ %
  \ifdim\Gin@nat@width>\linewidth
    \linewidth
  \else
    \Gin@nat@width
  \fi
}
\makeatother

\usepackage{Sweavel}


\input{./everything/everything.tex}
% for the mathscr
\usepackage{mathrsfs}
% easier for bold symbols
\newcommand{\bs}[1]{\boldsymbol{#1}}
% partial derivatives
\newcommand{\partiald}[1]{\frac{\delta}{\delta#1}}
% second partial derivative
\newcommand{\partialdtwo}[1]{\frac{\delta^2}{\delta#1^2}}
% for estimators
\newcommand{\wh}[1]{\widehat{#1}}
% for examples
\newcommand{\gb}[1]{\greybox{#1}}
% text over arrow
\usepackage{mathtools}
% command for symbols under symbols
\newcommand{\under}[2]{\mathop{#1}\limits_{#2}}
% cancel to zero
\usepackage{cancel}
% Example: \cancelto{0}{x}

\titleformat{\section}{\normalfont\Large\bfseries}{}{0pt}{}
% for sets and curly letters
\newcommand{\cur}[1]{\mathcal{#1}}
\newcommand{\scr}[1]{\mathscr{#1}}

\usepackage{listings}
\definecolor{royalblue}{rgb}{0.25, 0.41, 0.88}
\definecolor{forestgreen}{rgb}{0.0, 0.27, 0.13}
\definecolor{rust}{rgb}{0.72, 0.25, 0.05}

\lstset{ 
  language=R,                     % the language of the code
  basicstyle=\footnotesize, % the size of the fonts that are used for the code
  numbers=left,                   % where to put the line-numbers
  numberstyle=\tiny\color{gray!150},  % the style that is used for the line-numbers
  stepnumber=1,                   % the step between two line-numbers. If it is 1, each line
                                  % will be numbered
  numbersep=5pt,                  % how far the line-numbers are from the code
  backgroundcolor=\color{gray!40},  % choose the background color. You must add \usepackage{color}
  showspaces=false,               % show spaces adding particular underscores
  showstringspaces=false,         % underline spaces within strings
  showtabs=false,                 % show tabs within strings adding particular underscores
  frame=single,                   % adds a frame around the code
  rulecolor=\color{black},        % if not set, the frame-color may be changed on line-breaks within not-black text (e.g. commens (green here))
  tabsize=2,                      % sets default tabsize to 2 spaces
  captionpos=b,                   % sets the caption-position to bottom
  breaklines=true,                % sets automatic line breaking
  breakatwhitespace=false,        % sets if automatic breaks should only happen at whitespace
  keywordstyle=\color{royalblue},      % keyword style
  commentstyle=\color{forestgreen},   % comment style
  stringstyle=\color{rust}      % string literal style
} 

\begin{document}
% refer to https://yihui.org/knitr/options/#code-decoration for more options


% (1) Remember that we want the default page style
% (2) but for this page we want an empty page style for the title 
% (3) we input our title 
% (4) we call a new page and reset the page counter to 1
\pagestyle{default}
\thispagestyle{empty}
\includegraphics[width=8cm]{./UsydLogo}

\vspace{1cm}


\horline
{\centering\bfseries \Large \textsc{STAT3023} Statiscal Inference

}
\horline

\vspace{3cm}

{\large \centering Lab Week 10

}

{\centering

\vspace{1cm}

Tutor: Wen Dai

SID: 470408326

\vspace{1cm}

School of Mathematics and Statistics

The University of Sydney

\vfill

Semester 2, 2021\newpage

}

\newpage
\setcounter{page}{1}
\begin{enumerate}[label={\bfseries\arabic*}]
\item 

  \begin{enumerate}[label=(\alph*)]
  \item We write a function that computes the MLE based interval
\begin{Schunk}
\begin{Sinput}
mle.int = function(x, C) {
    # x is a random sample:
    thetaML = 1/mean(x)
    # C is half with width of the intervals
    d1 = max(thetaML, C)
    c(d1 - C, d1 + C)
}
\end{Sinput}
\end{Schunk}
  \item We test it by generating a random sample:
\begin{Schunk}
\begin{Sinput}
n = 4
C = 1.5
th0 = 2
x = rexp(n, rate = th0)
mle_interval = mle.int(x, C)
mle_interval
\end{Sinput}
\begin{Soutput}
[1] 1.742083 4.742083
\end{Soutput}
\end{Schunk}
  \end{enumerate}

\item 
  \begin{enumerate}[label=(\alph*)]
  \item We now write a function called \verb+bayes.int()+ which computes the Bayes interval based on the flat prior.
\begin{Schunk}
\begin{Sinput}
bayes.int = function(x, C) {
    d2 = C * (exp(2 * mean(x) * C) + 1)/(exp(2 * mean(x) * C) - 1)
    c(d2 - C, d2 + C)
}
\end{Sinput}
\end{Schunk}
  \item Again we test it by testing the same sample
\begin{Schunk}
\begin{Sinput}
n = 4
C = 1.5
th0 = 2
x = rexp(n, rate = th0)
bayes_interval = bayes.int(x, C)
bayes_interval
\end{Sinput}
\begin{Soutput}
[1] 1.204336 4.204336
\end{Soutput}
\end{Schunk}
  \item We visualize the interval by plotting the posterior curve and interval on the same plot. 
\begin{Schunk}
\begin{Sinput}
m = mean(x)
# label the Origin
plot(0, 0, xlim = c(0, 3/m), ylim = c(0, 0.4), xlab = expression(theta), ylab = "probability")
curve(dgamma(x, shape = n + 1, rate = n * m), from = 0, to = 3/m, add = TRUE)
abline(v = mle_interval, col = "blue", lty = 2)
abline(v = bayes_interval, col = "red", lty = 2)
abline(h = dgamma(bayes_interval[1], shape = n + 1, rate = n * m), lty = 3)
legend("topright", legend = c("posterior density", "mle interval", "bayes interval"), col = c("black",
    "blue", "red"), lty = c(1, 2, 2))
\end{Sinput}


{\centering \includegraphics[width=\maxwidth]{figure/unnamed-chunk-6-1} 

}

\end{Schunk}
  \end{enumerate}
\item We plot the non-coverage probabilities of $\theta$ values for the MLE based interval vs the Bayes interval. 
\begin{Schunk}
\begin{Sinput}
th = (1:500)/50
L = length(th)
B = 1000
noncoverage.mle = noncoverage.bayes = 0
c = 1.5

# we fix the theta value
for (i in 1:L) {
    mle.mat = matrix(0, B, 2)
    bayes.mat = matrix(0, B, 2)

    # we fix the row in our matrix.
    for (j in 1:B) {
        x = rexp(4, rate = th[i])  # draw 4 random numbers from the exp distn. 
        mle.mat[j, ] = mle.int(x, c)  # constructing intervals
        bayes.mat[j, ] = bayes.int(x, c)
    }

    # count the number of intervals not containing theta
    noncoverage.mle[i] = sum(th[i] < mle.mat[, 1]) + sum(th[i] > mle.mat[, 2])
    noncoverage.bayes[i] = sum(th[i] < bayes.mat[, 1]) + sum(th[i] > bayes.mat[, 2])
}
plot(th, noncoverage.mle/B, type = "l", col = "blue", main = "Simulated proportions of non-coverage for different interval methods",
    xlab = expression(theta), ylab = "Empirical noncoverage proportion", cex.main = 0.85, cex.axis = 0.8)

lines(th, noncoverage.bayes/B, type = "l", col = "red")
legend("topleft", legend = c("mle", "Bayes flat prior"), col = c("blue", "red"), lty = c(1, 1))
\end{Sinput}


{\centering \includegraphics[width=\maxwidth]{figure/unnamed-chunk-7-1} 

}

\end{Schunk}
\item We compute the risk for the MLE based interval for each value of $\theta$ from our vector \verb+th+ and we save it in a vector called \verb+m.risk+. 

\begin{Schunk}
\begin{Sinput}
risk.overest = pgamma(q = 1/(th + c), shape = n, rate = n * th)  # this applies to all values in th
big = (th >= (2 * C))  # find the theta values >= 2C
risk.underest = 0 * risk.overest  # Start with a vector of zeroes
risk.underest[big] = pgamma(q = 1/(th[big] + c), shape = n, rate = n * th[big]) + pgamma(q = 1/(th[big] +
    c), shape = n, rate = n * th[big], lower.tail = FALSE)
m.risk = risk.overest + risk.underest
\end{Sinput}
\end{Schunk}
\item We compute the risk for the Bayes interval for each values of $\theta$ from our vector \verb+th+ and we save it in a vector called \verb+m.risk+. 
\begin{Schunk}
\begin{Sinput}
risk.overest = pgamma(q = 1/(2 * c) * log(1 + 2 * c/th), shape = n, rate = n * th)
risk.underest = 0 * risk.overest  # Start with a vector of zeroes
big = th >= 2 * c
risk.underest[big] = pgamma(q = 1/(2 * c) * log(1 + 2 * c/th[big]), shape = n, rate = n * th[big]) +
    pgamma(q = 1/(2 * c) * log(th[big]/(th[big] - 2 * c)), shape = n, rate = n * th[big], lower.tail = FALSE)
b.risk = risk.overest + risk.underest
\end{Sinput}
\end{Schunk}
We now plot our curves:
\begin{Schunk}
\begin{Sinput}
plot(th, m.risk, col = "blue", lwd = 0.2, main = "risk of mle interval and bayes interval for different theta values",
    cex.main = 0.85, cex.axis = 0.8, xlab = expression(theta), ylab = "risk")
lines(th, b.risk, col = "red", lwd = 2)
legend("bottomright", legend = c("mle", "Bayes flat prior"), col = c("blue", "red"), lty = c(1, 1))
\end{Sinput}


{\centering \includegraphics[width=\maxwidth]{figure/unnamed-chunk-10-1} 

}

\end{Schunk}
The values of $\theta$ for which the mle interval does better are the $\theta$ values \tbf{less than} 3. The interval in which the bayes interval does better are the $\theta$ values \tbf{greater than} 3. 

The two intervals have somewhat similar performances for $\theta < 3$ but the bayes interval is noticably better for $\theta > 3$. 

Near the value of $\theta = 3$ there is a discontinuity for both the mle and the bayes intervals. This is because of the fact that the intervals are different depending on if you're in the region of $0 < \theta < 2C$ or the region $\theta > 2C$. In our example, $C = 1.5$ and hence the intervals are drastically different on the boundary of $2C = 2(1.5) = 3$
\end{enumerate}

\end{document}
