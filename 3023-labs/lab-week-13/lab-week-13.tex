\documentclass[12pt, a4paper]{article}\usepackage[]{graphicx}\usepackage[]{color}
% maxwidth is the original width if it is less than linewidth
% otherwise use linewidth (to make sure the graphics do not exceed the margin)
\makeatletter
\def\maxwidth{ %
  \ifdim\Gin@nat@width>\linewidth
    \linewidth
  \else
    \Gin@nat@width
  \fi
}
\makeatother

\usepackage{Sweavel}


\input{./everything/everything.tex}
% for the mathscr
\usepackage{mathrsfs}
% easier for bold symbols
\newcommand{\bs}[1]{\boldsymbol{#1}}
% partial derivatives
\newcommand{\partiald}[1]{\frac{\delta}{\delta#1}}
% second partial derivative
\newcommand{\partialdtwo}[1]{\frac{\delta^2}{\delta#1^2}}
% for estimators
\newcommand{\wh}[1]{\widehat{#1}}
% for examples
\newcommand{\gb}[1]{\greybox{#1}}
% text over arrow
\usepackage{mathtools}
% command for symbols under symbols
\newcommand{\under}[2]{\mathop{#1}\limits_{#2}}
% cancel to zero
\usepackage{cancel}
% Example: \cancelto{0}{x}

\titleformat{\section}{\normalfont\Large\bfseries}{}{0pt}{}
% for sets and curly letters
\newcommand{\cur}[1]{\mathcal{#1}}
\newcommand{\scr}[1]{\mathscr{#1}}

\usepackage{listings}
\definecolor{royalblue}{rgb}{0.25, 0.41, 0.88}
\definecolor{forestgreen}{rgb}{0.0, 0.27, 0.13}
\definecolor{rust}{rgb}{0.72, 0.25, 0.05}

\lstset{ 
  language=R,                     % the language of the code
  basicstyle=\footnotesize, % the size of the fonts that are used for the code
  numbers=left,                   % where to put the line-numbers
  numberstyle=\tiny\color{gray!150},  % the style that is used for the line-numbers
  stepnumber=1,                   % the step between two line-numbers. If it is 1, each line
                                  % will be numbered
  numbersep=5pt,                  % how far the line-numbers are from the code
  backgroundcolor=\color{gray!40},  % choose the background color. You must add \usepackage{color}
  showspaces=false,               % show spaces adding particular underscores
  showstringspaces=false,         % underline spaces within strings
  showtabs=false,                 % show tabs within strings adding particular underscores
  frame=single,                   % adds a frame around the code
  rulecolor=\color{black},        % if not set, the frame-color may be changed on line-breaks within not-black text (e.g. commens (green here))
  tabsize=2,                      % sets default tabsize to 2 spaces
  captionpos=b,                   % sets the caption-position to bottom
  breaklines=true,                % sets automatic line breaking
  breakatwhitespace=false,        % sets if automatic breaks should only happen at whitespace
  keywordstyle=\color{royalblue},      % keyword style
  commentstyle=\color{forestgreen},   % comment style
  stringstyle=\color{rust}      % string literal style
} 

\begin{document}
% refer to https://yihui.org/knitr/options/#code-decoration for more options


% (1) Remember that we want the default page style
% (2) but for this page we want an empty page style for the title 
% (3) we input our title 
% (4) we call a new page and reset the page counter to 1
\pagestyle{default}
\thispagestyle{empty}
\includegraphics[width=8cm]{./UsydLogo}

\vspace{1cm}


\horline
{\centering\bfseries \Large \textsc{STAT3023} Statiscal Inference

}
\horline

\vspace{3cm}

{\large \centering Lab Week 10

}

{\centering

\vspace{1cm}

Tutor: Wen Dai

SID: 470408326

\vspace{1cm}

School of Mathematics and Statistics

The University of Sydney

\vfill

Semester 2, 2021\newpage

}

\newpage
\setcounter{page}{1}
\tableofcontents
\newpage
\section{Lab 7}
\begin{enumerate}[label={\bfseries\arabic*.}]
  \item 
  \begin{enumerate}[label={(\alph*)}]
    \item We assume a null hypothesis of $\theta = \theta_0 = 1$. We are given that $gamma_0 = 1$. 
\begin{Schunk}
\begin{Sinput}
alpha = 0.05
# find a and b
a = qgamma(p = alpha/2, shape = 1, scale = 1, lower.tail = TRUE)
b = qgamma(p = alpha/2, shape = 1, scale = 1, lower.tail = FALSE)
c(a, b)
\end{Sinput}
\begin{Soutput}
[1] 0.02531781 3.68887945
\end{Soutput}
\end{Schunk}
  \item We plot the power function of the equal tailed test
\begin{Schunk}
\begin{Sinput}
# Define a vector of theta-values
th = (250:1500)/1000
# obtain a corresponding vector of values of the power
power_level = pgamma(q = a, shape = 1, scale = th, lower.tail = TRUE) + pgamma(q = b, shape = 1, scale = th,
    lower.tail = FALSE)
# Plot power against theta
plot(th, power_level, xlab = expression(theta), ylab = "power", main = "power of level test", lwd = 0.5)
# dashed line
abline(h = 0.05, lty = 2)
\end{Sinput}


{\centering \includegraphics[width=\maxwidth]{figure/unnamed-chunk-3-1} 

}

\end{Schunk}
\item We write a function which takes a level $\alpha$ and returns the elements $c$ and $d$ for the UMPU test. We do this for $\theta_0 = 1$ and $\alpha = 0.05$
  \end{enumerate}
\end{enumerate}


\end{document}
