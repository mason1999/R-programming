\documentclass[12pt, a4paper]{article}\usepackage[]{graphicx}\usepackage[]{color}
% maxwidth is the original width if it is less than linewidth
% otherwise use linewidth (to make sure the graphics do not exceed the margin)
\makeatletter
\def\maxwidth{ %
  \ifdim\Gin@nat@width>\linewidth
    \linewidth
  \else
    \Gin@nat@width
  \fi
}
\makeatother

\definecolor{fgcolor}{rgb}{0.345, 0.345, 0.345}
\newcommand{\hlnum}[1]{\textcolor[rgb]{0.686,0.059,0.569}{#1}}%
\newcommand{\hlstr}[1]{\textcolor[rgb]{0.192,0.494,0.8}{#1}}%
\newcommand{\hlcom}[1]{\textcolor[rgb]{0.678,0.584,0.686}{\textit{#1}}}%
\newcommand{\hlopt}[1]{\textcolor[rgb]{0,0,0}{#1}}%
\newcommand{\hlstd}[1]{\textcolor[rgb]{0.345,0.345,0.345}{#1}}%
\newcommand{\hlkwa}[1]{\textcolor[rgb]{0.161,0.373,0.58}{\textbf{#1}}}%
\newcommand{\hlkwb}[1]{\textcolor[rgb]{0.69,0.353,0.396}{#1}}%
\newcommand{\hlkwc}[1]{\textcolor[rgb]{0.333,0.667,0.333}{#1}}%
\newcommand{\hlkwd}[1]{\textcolor[rgb]{0.737,0.353,0.396}{\textbf{#1}}}%
\let\hlipl\hlkwb

\usepackage{framed}
\makeatletter
\newenvironment{kframe}{%
 \def\at@end@of@kframe{}%
 \ifinner\ifhmode%
  \def\at@end@of@kframe{\end{minipage}}%
  \begin{minipage}{\columnwidth}%
 \fi\fi%
 \def\FrameCommand##1{\hskip\@totalleftmargin \hskip-\fboxsep
 \colorbox{shadecolor}{##1}\hskip-\fboxsep
     % There is no \\@totalrightmargin, so:
     \hskip-\linewidth \hskip-\@totalleftmargin \hskip\columnwidth}%
 \MakeFramed {\advance\hsize-\width
   \@totalleftmargin\z@ \linewidth\hsize
   \@setminipage}}%
 {\par\unskip\endMakeFramed%
 \at@end@of@kframe}
\makeatother

\definecolor{shadecolor}{rgb}{.97, .97, .97}
\definecolor{messagecolor}{rgb}{0, 0, 0}
\definecolor{warningcolor}{rgb}{1, 0, 1}
\definecolor{errorcolor}{rgb}{1, 0, 0}
\newenvironment{knitrout}{}{} % an empty environment to be redefined in TeX

\usepackage{alltt}
\input{./everything/everything.tex}
\IfFileExists{upquote.sty}{\usepackage{upquote}}{}
\begin{document}
% refer to https://yihui.org/knitr/options/#code-decoration for more options


% (1) Remember that we want the default page style
% (2) but for this page we want an empty page style for the title 
% (3) we input our title 
% (4) we call a new page and reset the page counter to 1
\pagestyle{default}
\thispagestyle{empty}
\includegraphics[width=8cm]{./UsydLogo}

\vspace{1cm}


\horline
{\centering\bfseries \Large \textsc{STAT3023} Statiscal Inference

}
\horline

\vspace{3cm}

{\large \centering Lab Week 10

}

{\centering

\vspace{1cm}

Tutor: Wen Dai

SID: 470408326

\vspace{1cm}

School of Mathematics and Statistics

The University of Sydney

\vfill

Semester 2, 2021\newpage

}

\newpage
\setcounter{page}{1}
\begin{enumerate}
  \item 
  \begin{enumerate}[label=(\alph*)]
  \item We generate 1000 samples, sampling from a binomial distribution with size being 200 and $p = 0.3$
  
\begin{knitrout}\small
\definecolor{shadecolor}{rgb}{0.969, 0.969, 0.969}\color{fgcolor}\begin{kframe}
\begin{alltt}
\hlstd{samples} \hlkwb{=} \hlnum{1000}
\hlstd{size} \hlkwb{=} \hlnum{200}
\hlstd{p} \hlkwb{=} \hlnum{0.3}
\hlstd{X} \hlkwb{=} \hlkwd{rbinom}\hlstd{(samples, size, p)}
\hlkwd{hist}\hlstd{(X,} \hlkwc{prob} \hlstd{=} \hlnum{TRUE}\hlstd{,} \hlkwc{main} \hlstd{=} \hlstr{"Distribution of the bin(200, 0.3)"}\hlstd{)}
\hlstd{X_mean} \hlkwb{=} \hlstd{size} \hlopt{*} \hlstd{p}
\hlstd{X_sd} \hlkwb{=} \hlkwd{sqrt}\hlstd{(size} \hlopt{*} \hlstd{p} \hlopt{*} \hlstd{(}\hlnum{1} \hlopt{-} \hlstd{p))}
\hlkwd{curve}\hlstd{(}\hlkwd{dnorm}\hlstd{(x,} \hlkwc{mean} \hlstd{= X_mean,} \hlkwc{sd} \hlstd{= X_sd),} \hlkwc{xlim} \hlstd{=} \hlkwd{c}\hlstd{(}\hlnum{0}\hlstd{,} \hlnum{200}\hlstd{),} \hlkwc{add} \hlstd{=} \hlnum{TRUE}\hlstd{)}
\end{alltt}
\end{kframe}

{\centering \includegraphics[width=\maxwidth]{figure/unnamed-chunk-2-1} 

}


\end{knitrout}
  
  \item 
  We find the $P(45 \le X \le 55)$ 
\begin{knitrout}\small
\definecolor{shadecolor}{rgb}{0.969, 0.969, 0.969}\color{fgcolor}\begin{kframe}
\begin{alltt}
\hlkwd{pbinom}\hlstd{(}\hlnum{55}\hlstd{, size, p)} \hlopt{-} \hlkwd{pbinom}\hlstd{(}\hlnum{45}\hlstd{, size, p)}
\end{alltt}
\begin{verbatim}
[1] 0.2343248
\end{verbatim}
\end{kframe}
\end{knitrout}
  \item  We use the normal approximation to the binomial with mean $np$ and standard deviation $np(1-p)$ to approximate the same probability. With the continuity correction, our probability becomes $$P(45 \le X \le 55) = P(44.5 \le X \le 55.5)$$
\begin{knitrout}\small
\definecolor{shadecolor}{rgb}{0.969, 0.969, 0.969}\color{fgcolor}\begin{kframe}
\begin{alltt}
\hlkwd{pnorm}\hlstd{(}\hlnum{55.5}\hlstd{,} \hlkwc{mean} \hlstd{= X_mean,} \hlkwc{sd} \hlstd{= X_sd)} \hlopt{-} \hlkwd{pnorm}\hlstd{(}\hlnum{45.5}\hlstd{,} \hlkwc{mean} \hlstd{= X_mean,} \hlkwc{sd} \hlstd{= X_sd)}
\end{alltt}
\begin{verbatim}
[1] 0.2310965
\end{verbatim}
\end{kframe}
\end{knitrout}
  \end{enumerate}
 \item  
  \begin{enumerate}
  \item
  We simulate drawing 1000 times from the binomial and the poisson distribution here:
\begin{knitrout}\small
\definecolor{shadecolor}{rgb}{0.969, 0.969, 0.969}\color{fgcolor}\begin{kframe}
\begin{alltt}
\hlstd{simulations} \hlkwb{=} \hlnum{1000}
\hlstd{size} \hlkwb{=} \hlnum{200}
\hlstd{p} \hlkwb{=} \hlnum{0.03}
\hlstd{lambda} \hlkwb{=} \hlnum{6}

\hlcom{# 1000 draws of a binomial distribution}
\hlstd{X} \hlkwb{=} \hlkwd{rbinom}\hlstd{(simulations, size, p)}
\hlcom{# 1000 draws of a poisson distribution}
\hlstd{Y} \hlkwb{=} \hlkwd{rpois}\hlstd{(simulations, lambda)}
\hlcom{# plot with 1 row and 2 columns}
\hlkwd{par}\hlstd{(}\hlkwc{mfrow}\hlstd{=}\hlkwd{c}\hlstd{(}\hlnum{1}\hlstd{,}\hlnum{2}\hlstd{))}
\hlcom{# Historgrams of X and Y}
\hlkwd{hist}\hlstd{(X,} \hlkwc{probability} \hlstd{=} \hlnum{TRUE}\hlstd{,} \hlkwc{main} \hlstd{=} \hlstr{"binomial distribution"}\hlstd{,} \hlkwc{breaks} \hlstd{=} \hlnum{15}\hlstd{)}
\hlkwd{hist}\hlstd{(Y,} \hlkwc{probability} \hlstd{=} \hlnum{TRUE}\hlstd{,} \hlkwc{main} \hlstd{=} \hlstr{"Poisson distribution"}\hlstd{,} \hlkwc{breaks} \hlstd{=} \hlnum{15}\hlstd{)}
\end{alltt}
\end{kframe}

{\centering \includegraphics[width=\maxwidth]{figure/unnamed-chunk-5-1} 

}


\end{knitrout}
  \item We find $P(X \le 5)$ and $P(Y \le 5)$ 
\begin{knitrout}\small
\definecolor{shadecolor}{rgb}{0.969, 0.969, 0.969}\color{fgcolor}\begin{kframe}
\begin{alltt}
  \hlkwd{pbinom}\hlstd{(}\hlnum{5}\hlstd{,} \hlkwc{size} \hlstd{=} \hlnum{200}\hlstd{,} \hlkwc{prob} \hlstd{=} \hlnum{0.03}\hlstd{)}
\end{alltt}
\begin{verbatim}
[1] 0.4432292
\end{verbatim}
\begin{alltt}
  \hlkwd{ppois}\hlstd{(}\hlnum{5}\hlstd{,} \hlkwc{lambda} \hlstd{=} \hlnum{6}\hlstd{)}
\end{alltt}
\begin{verbatim}
[1] 0.4456796
\end{verbatim}
\end{kframe}
\end{knitrout}
  \end{enumerate}
  \item Recall that for $n$ iid random variables $X_1, ..., X_n$ the standardised sum is given by:
  $$S = \dfrac{\sum_{i = 1}^n\brac{X_i - \E{x_i}}}{\sqrt{n\Var{X_1}}}$$
  \begin{enumerate}
    \item Generate 1000 realisations of the sum of $n$ (where $n = 5$) from the uniform distirbution $unif(0, 1)$. Note that here, $\E{X_i} = \frac{1}{2}$ and that $\Var{X_i} = \frac{1}{12}$. Then do the same for $n = 100$ and caclulate the standardised residuals for both. 
    
    
\begin{knitrout}\small
\definecolor{shadecolor}{rgb}{0.969, 0.969, 0.969}\color{fgcolor}\begin{kframe}
\begin{alltt}
\hlkwd{set.seed}\hlstd{(}\hlnum{3023}\hlstd{)}
\hlstd{sim_number} \hlkwb{=} \hlnum{1000}
\hlstd{n1} \hlkwb{=} \hlnum{5}
\hlstd{n2} \hlkwb{=} \hlnum{100}
\hlstd{mu} \hlkwb{=} \hlnum{1}\hlopt{/}\hlnum{2}
\hlstd{sigma_squared} \hlkwb{=} \hlnum{1}\hlopt{/}\hlnum{12}

\hlcom{# sample from the uniform distribution n1 times and repeat the experiment }
\hlcom{# sim_number of times}
\hlstd{temp1} \hlkwb{=} \hlstd{(}\hlkwd{runif}\hlstd{(sim_number} \hlopt{*} \hlstd{n1)} \hlopt{-} \hlstd{mu)}\hlopt{/}\hlkwd{sqrt}\hlstd{(n1} \hlopt{*} \hlstd{sigma_squared)}
\hlstd{mat_temp1} \hlkwb{=} \hlkwd{matrix}\hlstd{(temp1,} \hlkwc{ncol} \hlstd{= n1)}
\hlkwd{dim}\hlstd{(mat_temp1)}
\end{alltt}
\begin{verbatim}
[1] 1000    5
\end{verbatim}
\begin{alltt}
\hlstd{S1} \hlkwb{=} \hlkwd{apply}\hlstd{(mat_temp1,} \hlnum{1}\hlstd{, sum)}

\hlstd{temp2} \hlkwb{=} \hlstd{(}\hlkwd{runif}\hlstd{(sim_number} \hlopt{*} \hlstd{n2)} \hlopt{-} \hlstd{mu)}\hlopt{/}\hlkwd{sqrt}\hlstd{(n2} \hlopt{*} \hlstd{sigma_squared)}
\hlstd{mat_temp2} \hlkwb{=} \hlkwd{matrix}\hlstd{(temp2,} \hlkwc{ncol} \hlstd{= n2)}
\hlkwd{dim}\hlstd{(mat_temp2)}
\end{alltt}
\begin{verbatim}
[1] 1000  100
\end{verbatim}
\begin{alltt}
\hlstd{S2} \hlkwb{=} \hlkwd{apply}\hlstd{(mat_temp2,} \hlnum{1}\hlstd{, sum)}
\end{alltt}
\end{kframe}
\end{knitrout}
    
    
\begin{knitrout}\small
\definecolor{shadecolor}{rgb}{0.969, 0.969, 0.969}\color{fgcolor}\begin{kframe}
\begin{alltt}
\hlkwd{par}\hlstd{(}\hlkwc{mfrow} \hlstd{=} \hlkwd{c}\hlstd{(}\hlnum{1}\hlstd{,} \hlnum{2}\hlstd{))}
\hlkwd{hist}\hlstd{(S1,} \hlkwc{probability} \hlstd{=} \hlnum{TRUE}\hlstd{)}
\hlkwd{curve}\hlstd{(}\hlkwd{dnorm}\hlstd{(x),} \hlkwc{add} \hlstd{=} \hlnum{TRUE}\hlstd{)}
\hlkwd{hist}\hlstd{(S2,} \hlkwc{probability} \hlstd{=} \hlnum{TRUE}\hlstd{)}
\hlkwd{curve}\hlstd{(}\hlkwd{dnorm}\hlstd{(x),} \hlkwc{add} \hlstd{=} \hlnum{TRUE}\hlstd{)}
\end{alltt}
\end{kframe}

{\centering \includegraphics[width=\maxwidth]{figure/unnamed-chunk-8-1} 

}


\end{knitrout}
    
    \vfill
    
  \end{enumerate}
\end{enumerate}
\end{document}
